%%%%%%%%%%%%%%%%%%%%%%%%%%%%%%%%%%%%%%%%%
% Thin Sectioned Essay
% LaTeX Template
% Version 1.0 (3/8/13)
%
% This template has been downloaded from:
% http://www.LaTeXTemplates.com
%
% Original Author:
% Nicolas Diaz (nsdiaz@uc.cl) with extensive modifications by:
% Vel (vel@latextemplates.com)
%
% License:
% CC BY-NC-SA 3.0 (http://creativecommons.org/licenses/by-nc-sa/3.0/)
%
%%%%%%%%%%%%%%%%%%%%%%%%%%%%%%%%%%%%%%%%%

%----------------------------------------------------------------------------------------
%	PACKAGES AND OTHER DOCUMENT CONFIGURATIONS
%----------------------------------------------------------------------------------------

\documentclass[a4paper, 11pt]{article} % Font size (can be 10pt, 11pt or 12pt) and paper size (remove a4paper for US letter paper)

\usepackage{standalone}

\usepackage[a4paper,left=3cm,right=3cm]{geometry}

\usepackage{indentfirst}
%\usepackage[utf8]{inputenc}
\usepackage{pgfplots}
\usepackage{color}

\usepackage{enumerate}

\usepackage[protrusion=true,expansion=true]{microtype} % Better typography
\usepackage{graphicx} % Required for including pictures
\usepackage{wrapfig} % Allows in-line images

\usepackage{tikz}
\usetikzlibrary{shapes}
\usepackage{pgf}
\usepackage{pgfplots}
\usetikzlibrary{arrows}
\usepackage{tkz-graph}
\usetikzlibrary{shapes.multipart}

\renewcommand{\thesection}{\Roman{section}}
\usepackage{titlesec}
%\titlespacing\section{0pt}{20pt plus 4pt minus 2pt}{-5pt plus 2pt minus 2pt}

\usepackage{listings}
\usepackage{textcomp}
\usepackage{xcolor}
\lstset{
language=Java,
basicstyle=\normalsize, % ou \c ca==> basicstyle=\scriptsize,
upquote=true,
aboveskip={1.2\baselineskip},
columns=fullflexible,
showstringspaces=false,
extendedchars=true,
breaklines=true,
showtabs=false,
showspaces=false,
showstringspaces=false,
identifierstyle=\ttfamily,
keywordstyle=\color[rgb]{0,0,1},
commentstyle=\color[rgb]{0.133,0.545,0.133},
stringstyle=\color[rgb]{0.627,0.126,0.941},
}

\lstset{% This applies to ALL lstlisting
    backgroundcolor=\color{yellow!10},%
    numbers=left, numberstyle=\tiny, stepnumber=2, numbersep=5pt,%
    }%

% Applies only when you use it
\lstdefinestyle{MyLang}{
    basicstyle=\small\ttfamily\color{magenta},%
    breaklines=true,%                                      allow line breaks
    moredelim=[s][\color{green!50!black}\ttfamily]{'}{'},% single quotes in green
    moredelim=*[s][\color{black}\ttfamily]{options}{\}},%  options in black (until trailing })
    commentstyle={\color{gray}\itshape},%                  gray italics for comments
    morecomment=[l]{//},%                                  define // comment
    emph={%
        STRING%                                            literal strings listed here
        },emphstyle={\color{blue}\ttfamily},%              and formatted in blue
    alsoletter={:,|,;},%
    morekeywords={:,|,;},%                                 define the special characters
    keywordstyle={\color{black}},%                         and format them in black
}

\lstdefinestyle{Pyth}{
	language=Python
}


\usepackage{mathpazo} % Use the Palatino font
\usepackage[T1]{fontenc} % Required for accented characters
\linespread{1.05} % Change line spacing here, Palatino benefits from a slight increase by default

\makeatletter
\renewcommand\@biblabel[1]{\textbf{#1.}} % Change the square brackets for each bibliography item from '[1]' to '1.'
\renewcommand{\@listI}{\itemsep=0pt} % Reduce the space between items in the itemize and enumerate environments and the bibliography

\renewcommand{\maketitle}{ % Customize the title - do not edit title and author name here, see the TITLE block below
\begin{flushright} % Right align
{\LARGE\@title} % Increase the font size of the title

\vspace{50pt} % Some vertical space between the title and author name

{\large\@author} % Author name
\\\@date % Date

\vspace{-20pt} % Some vertical space between the author block and abstract
\end{flushright}
}

%----------------------------------------------------------------------------------------
%	TITLE
%----------------------------------------------------------------------------------------

\title{\textbf{Introduction aux Syst�mes Multi-Agents}\\ % Title
Projet : SimTransport} % Subtitle

\author{\textsc{Fran\c cois Hernandez - L\'eo Pons} % Author
\\{\textit{CentraleSup\'elec}}} % Institution

\date{\today} % Date

%----------------------------------------------------------------------------------------

\begin{document}

\maketitle % Print the title section

\pagebreak
\tableofcontents

\pagebreak

%----------------------------------------------------------------------------------------
%	ABSTRACT AND KEYWORDS
%----------------------------------------------------------------------------------------

%\renewcommand{\abstractname}{Summary} % Uncomment to change the name of the abstract to something else



\vspace{30pt} % Some vertical space between the abstract and first section

%----------------------------------------------------------------------------------------
%	ESSAY BODY
%----------------------------------------------------------------------------------------

\section*{Introduction}
\addcontentsline{toc}{section}{Introduction}

Un Syst�me Multi-Agents est un syst�me compos� d'un ensemble d'agents, situ�s dans un certain environnement et interagissant selon certaines relations. Un agent est une entit� caract�ris�e par le fait qu'elle est, au moins partiellement, autonome. Le fonctionnement du syst�me est d�fini par le fonctionnement des agents, le syst�me central ne fait qu'office d'environnement d'�volution et de communication des diff�rents agents. On parle d'intelligence artificielle distribu�e.\\

Ce projet a pour objectif de construire une simulation des transports sur le plateau de Saclay. Pour cela, on utilisera la plateforme Jade. La plateforme Jade est une plateforme r�partie d'agents s'ex�cutant de mani�re asynchrone (chaque agent est une thread). Chaque agent d�finit ses actions � l'aide de comportements qu'il ex�cute les uns apr�s les autres. La programmation de ces comportements se fait de mani�re � les "entrelacer' dans l'ex�cution de l'agent. Dans ces comportements, les agents peuvent envoyer et recevoir des messages, mais ils peuvent aussi d�clencher d'autre comportements ou m�me cr�er d'autres agents.\\


%------------------------------------------------
\pagebreak
\section{Impl�mentation}

\subsection{Objets et structure du programme}

\subsection{Agents et comportements}

\subsection{Interface graphique}


%------------------------------------------------
\pagebreak

\section{Principaux param�tres de la simulation}

%------------------------------------------------
\pagebreak

\section{Simulation de diff�rentes politiques d'am�nagement}


%------------------------------------------------

%\pagebreak
%\section*{Conclusion}
%\addcontentsline{toc}{section}{Conclusion}


%----------------------------------------------------------------------------------------

\end{document}